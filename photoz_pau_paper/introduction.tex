In this \doctype \ we present the study of the photo-$z$ performance and its impact on clustering measurements expected in the PAU survey in a sample consisting of all galaxies of all types with $i_{AB} < 22.5$. Based of detailed simulation studies~\citep{Gaztanaga2012}, the requirement for the precision is set at $\sigma(z)/(1+z) = 0.0035$. The PAU survey will observe many galaxies beyond the $i_{AB}=22.5$ limit that play a crucial role in the PAU science case~\citep{Gaztanaga2012}. We will also study the performance in a fainter galaxy sample with $22.5 < i_{AB} \lesssim 23.7$, expecting to reach a photo-$z$ precision not worse than $\sigma(z)/(1+z) = 0.05$~\citep{Gaztanaga2012}. Finally, we will also study the impact on the photo-$z$ performance of small variations on the default filter set, in terms of width, wavelength coverage, etc.

Throughout the \doctype\ we will be using the Bayesian Photo-Z ({\tt BPZ}) template-based code from~\citet{Benitez2000}, after adapting it to our needs. We have also tried several photo-$z$ codes based on training methods. We have found that, because of the large, ${\cal O}(50)$, number of filters, some of them run into difficulties due to the combinatorial growth of the complexity of the problem, while others confirm the results presented here. The results obtained with training methods will be described in detail elsewhere (Bonnett et al., in preparation).

The outline of the \doctype\ is as follows. In section~\ref{sec:filt} we present the default PAU filter set. Section~\ref{sec:mock} discusses the mock galaxy samples that we use in our study, the noise generation, and the split into a bright and a faint galaxy samples. In section~\ref{sec:photoz} we introduce the {\tt BPZ} original code and our modifications, with special emphasis on the prior redshift probability distributions and the {\em odds} parameter. We also show the results obtained when running {\tt BPZ} on the mock catalog using the default filter set. Furthermore, we compute the so-called migration matrix $r_{ij}$ \citep{Gaztanaga2012}, corresponding to the probability that a galaxy at a true redshift bin $j$ is actually measured at a photo-$z$ bin $i$, and its effect on the measurement of galaxy auto- and cross-correlations. In section~\ref{sec:opti} we try several modifications to the filter set (wider/narrower filters, bluer/redder filters, etc.), study their performance on the brighter and fainter galaxy samples and find the optimal set. Finally, in section~\ref{sec:discussion}, we discuss the results and offer some conclusions.

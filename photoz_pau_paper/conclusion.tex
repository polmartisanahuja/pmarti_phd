\section{Discussion and Conclusions}
\label{sec:discussion}
In the previous sections, we have seen that, at the level of simulated data, a photo-$z$ precision of $\sigma_z \sim 0.0035(1+z)$ can be achieved for $\sim$50\% of galaxies at $i_{AB}<22.5$ by using a photometric filter system of 40 narrow bands of 125\AA \ width together with the \textit{ugrizY} broad bands. The precision degrades to $\sigma_z \sim 0.05(1+z)$ when we move to the magnitude range $22.5<i_{AB}<23.7$. These coincide with the two photo-$z$ precision requirements defined in \citet{Gaztanaga2012} needed to simultaneously measure Redshift Space Distortions (RSD) and Magnifications bias (MAG) on two samples, one on the foreground and one on the background, over the same area of the sky. The galaxies removed are the ones with the worst photo-$z$ quality according to our photo-$z$ algorithm used. In \paperodds\ it is shown that this kind of cuts, when they remove a substantial fraction of galaxies, can grossly bias the measured galaxy clustering. However, in the same \doctype\ \we\ propose a way to correct for it. 

On the other hand, we found that spiral galaxies are the ones that give the worst photo-$z$ performance. Moreover, quality cuts mostly remove them. Contrary to what was assumed in \cite{Benitez2009}, elliptical galaxies do not provide the best photo-$z$ performance, but irregular galaxies with prominent emission lines at $\sim$3737\AA \ [OII] and $\sim$5000\AA \ [OIII] are actually the ones that give the best performance. A possibility is that these two emission lines are better traced by the narrow bands than a single feature as the 4000\AA \ break of elliptical galaxies, making the photo-$z$ determination more robust. 

We also studied the effect of including the 40 narrow bands in a typical broad band filter set \textit{ugrizY}. We find that the $\Delta z / (1+z_{tr})$ distributions become more peaky around the maximum moving away from Gaussianity. Bias improves by an order of magnitude. Precision also improves in a factor of $\sim$5 below $i_{AB}\sim22.5$. However, the low signal-to-noise in narrow bands makes the improvement very small within $22.5<i_{AB}<23.7$, concluding that narrow bands at faint magnitudes are useful to improve the bias but not the precision.

We have also estimated the photo-$z$ migration matrices $r_{ij}$ which correspond to the probability
 that a galaxy observed at the photo-$z$ bin $i$ is actually at the true redshift bin $j$.
These are shown  in Fig.~\ref{plot:rij} for both the BS and FS.  We then show (in Fig.~\ref{plot:wij})
how this photo-$z$ migration matrix $r$ distorts the observed auto and cross-correlation of galaxies in narrow redshift bins.
 We show results for both the intrinsic clustering, which dominates the diagonal in the measured angular 
cross-correlation matrix,  $\bar{\omega}$, and the magnification effect, which appears as a diffused
off-diagonal cloud in $\bar{\omega}_{ij}$. The true cross-correlation matrix $\omega_{ij}$ 
can be obtained from the inverse migration $r^{-1}$  with a simple matrix operation: 
$\omega = r^{-1}\cdot\bar{\omega}\cdot(r^T)^{-1}$. This is a standard deconvolution problem, and it is only limited
by how well we know the migration matrix.

We find that introducing slight variations on the 40 narrow band filter set, such as: shifting bands to higher/lower wavelengths, narrowing/broadening or increasing logarithmically band widths, do not introduce significant changes on the final photo-$z$ performance.  Even so, general trends are that narrowing/broadening bands improves/worsens the photo-$z$ performance below/above $i_{AB}\sim22.5$, while red/blueshifting bands improve the photo-$z$ performance at high/low redshifts and the quality-cuts efficiency for early/late-type galaxies. Logarithmically growing band widths do not turn into any measurable improvement. Therefore, we conclude that the initial proposed filter set of 40 narrow bands seems to be close to optimal for the purposes of the PAU Survey at the WHT.

Additionally, we have also tried doubling the exposure times in all bands. The magnitude limits become $\sim0.5$~mag deeper, so that the new bright sample goes down to $i_{AB}<23$ and the new faint sample down to $i_{AB}<24.1$. Within these magnitude limits, the photo-$z$ performance remains almost unchanged, and we still reach the requirement in $\sigma_{68}$ after removing $\sim50$\% of the galaxies in both the new bright and the new faint catalogs. On the other hand, if we stick to the initial magnitude limits while still doubling the exposure times, we find that in this case the requirements are reached by removing only 30\% of the galaxies in the bright sample and none in the faint.

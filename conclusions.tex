Chapters~\ref{ch:pau} and \ref{ch:odds} conform the scientific work of this thesis and both have become two scientific publications, \citet{Marti2014b} and \citet{Marti2014a} respectively. 

In chapter~\ref{ch:pau} we have shown through simulated data that a photometric system composed of 40 narrow bands of 100~\AA \ width, as in the PAU@WHT survey~\citep{Benitez2009} camera (PAUCam), can deliver photometric redshifts with a precision of $\sigma_z \sim 0.0035(1+z)$ for the $\sim$50\% of galaxies with the best photo-$z$ quality up to magnitude $i_{AB}=22.5$. We have found that these galaxies are mainly ellipticals and irregulars. Contrary to what had been believed, prominent spectral emission lines ([OII] and [OIII]) in irregular galaxies make the resulting photo-$z$ performance even better than that obtained by tracking the 4000\AA \ break in elliptical galaxies. With such a photometric resolution, the photo-$z$ method becomes closer to the spectroscopic, where the position in wavelength of a single spectral line may be used to determine the redshift of the whole spectrum. 

We have also seen that broad-band filters ($ugrizY$), together with the narrow bands, help go deeper in magnitude. For $22.5<i_{AB}<23.7$, the photo-$z$ precision degrades by about an order of magnitude. 

Finally, we have studied how small deviations on the original narrow-band filters, in terms of width, wavelength coverage, etc., affect the photo-$z$ performances. We have found that results are very similar to the default, so we deduce that the original set is close to optimal.

PAUCam will see its first light in 2014, so real data will be available very soon leaving mock catalogs in the background. The photometric mock catalog used here is preliminary and relatively unsophisticated. The aim was just to have a first-order estimation of the PAU@WHT photo-$z$ performance. However, in the near future and in order to calibrate systematic effects, it would be very useful to have a more realistic mock catalog, including dust extinction on the galaxy photometry (reddening), adding more spectral type diversity (e.g. starburst galaxies, quasars or even stars), including evolution on the spectra, etc. Special emphasis should be given to increase the resolution of spectral templates, since template libraries for photo-$z$ determination \citep[e.g.][]{Coleman1980,Kinney1996} are intended to be used with broad-band filters ($\gtrsim$100~nm), so that narrow spectral features are omitted without affecting much the results. However, we have shown that spectral lines make a substantial difference when using narrow-bands ($\sim$10~nm). It would be interesting to study which spectral lines ($Ly_\alpha$, $H_\alpha$, $H_\beta$, [OII], [OIII], [OIII], etc.)
%with different ratios 
produce the better results. Furthermore, data could be simulated at the pixel level, so that the fluxes would be measured more realistically as well, including all possible systematics derived from the photometric extraction. 

Beyond this work, and as was said at the beginning of chapter~\ref{ch:pau}, we have also tried other methods, besides BPZ, for the PAU@WHT photo-$z$ determination: LePhare, which is also a template-fitting based method and some training-based methods (ANNz, ArborZ, random forest algorithms, nearest neighbors, etc). While template-fitting methods give very similar results, consistent with BPZ, the training methods give uneven results. Some of them are unable to cope with the combinatorial growth of the complexity of using a large number of filters (inputs), so they run into difficulties, while others are able to reproduce or even improve on the results presented here. 

In the fall of 2012, a new galaxy survey called the Dark Energy Survey (DES) saw its first light. When finished, it will have mapped about 5000~sq.~deg. of the southern sky up to a depth $i_{AB} \sim 24$, providing measurements of the positions in the sky, photometric redshifts and shapes of almost 300 milion galaxies up to redshift 1.2. During winter 2012/13, a Science Verification (SV) period of observations provided science-quality images for about 150~sq.~deg. In S\'anchez et al. (in preparation), the photo-$z$ performance of DES is studied using these data together with spectroscopic information of about 15000 galaxies from other surveys. If PAU@WHT surveys an area that overlaps the DES footprint, its quasi-spectroscopic information could be used as well to calibrate the DES photo-$z$s. 

In chapter~\ref{ch:odds} we have shown that applying photo-$z$ quality cuts on galaxy catalogs, similar to the ones applied in chapter~\ref{ch:pau} to achieve the required PAU@WHT photo-$z$ precision, can grossly bias the measured galaxy correlations within and across photometric redshift bins, since typically galaxies are not removed uniformly over the sky. Extending the work of \citet{Ho2012} and \citet{Ross2011}, we have developed a method to correct for this using the data themselves. We have applied it to the Mega-Z catalog~\citep{Collister2007}, containing $\sim$1 million luminous red galaxies in the redshift range $0.45<z<0.65$ and, after splitting the sample into four $\Delta z = 0.05$ photo-z bins using the BPZ algorithm, we have seen how the corrections bring the measured galaxy auto- and cross-correlations into agreement with expectations. 

In contrast to chapter~\ref{ch:pau}, the Mega-Z catalog only contains elliptical galaxies. Therefore, it would also be interesting to study how these cuts affect a galaxy catalog containing different spectral types. As shown in Figs.~\ref{bs_pz_results} and \ref{fs_pz_results}, different galaxy types provide different photo-$z$ quality. Since they also cluster with different amplitudes, the photo-$z$ quality cuts may bias the measured clustering. DES will provide photometry for different galaxy types with redshift up to $z\lesssim1.3$ and $i_{AB}\lesssim24$, which will make it very suitable to test if our correction also work in this case. Moreover, there are realistic DES mock catalogs that do not suffer from the systematic effects causing most of the extra clustering studied here, so they would help us disentangle the contributions due to observational systematics from those due to differences in the intrinsic clustering. 

We would like to note that although the correction has only been applied to the BAO-scale extraction, this may not be the best case to exemplify its usefulness. As shown in Fig.~\ref{Nz_bins}, photo-$z$ quality cuts are especially useful in order to reduce the long tails in the real redshift distribution within a given photo-$z$ bin. This reduces considerably the overlap between distant photo-$z$ bins, which is particularly useful in the measurement of weak-lensing magnification. This is because the contribution to the angular correlation of the terms due to the autocorrelation of overlapping galaxies at different photo-$z$ bins would be reduced with respect to the terms due to lensing, as can be deduced from Eq.~(\ref{eq:all4}).

Summarizing, in this work we have shown that, although photometric redshifts are typically considered as low-quality determinations of the real redshift, they can also perform reliably and precisely when a narrow-band filter set and photo-$z$ quality cuts are used. However, one must be careful when applying those cuts because the measured galaxy clustering can be highly biased due to either the inhomogeneous photometric quality over the sky or the different photo-$z$ quality associated with different galaxy types. We also provide a simple method to correct for this effect. We conclude that this work opens a new path to precise photometric redshift determinations in current and upcoming galaxy redshift surveys.

\chapter{Cosmological Framework}

\section{The geometry of the universe}

The cosmological principle asserts that the distribution of matter-energy in the universe at large scales is homogeneous and isotropic and, therefore, there is no prefered direcction in the sky.

According to the General Theory of Relativity, the unique metric compatible with that principle is the Friedmann-Lemaitre-Robertson-Walker (FLRW) metric, which space-time line element can be written as:  

\begin{equation}
ds^2 = -cdt^2 + a^2(t)dx^2
\end{equation}

The left-hand side $-ct$ of the sum is the temporal contribution while the right-hand side is the spatial, which is the product of an scale factor $a(t)$ and a comoving space line element $dx$. In the case of a perfect flat space, which seems to be in accord with the observations, and using spherical coordinates:

\begin{equation}
dx^2 \equiv dr^2 + r^2d\Omega^2
\end{equation}

\section{Cosmological Redshift}
When the light that comes from a galaxy travels across the space of an spanding universe sufers an effect called Cosmological Redshift. As the equivalent redshift due to the Doppler effect of a source in movement where the wavelength of the light is dilated, in the cosmological redshift the wavelength is also dilated but due to the dilatation of the space itself.

\begin{equation}
z \equiv {\lambda_{obs} - \lambda_{em} \over \lambda_{em}}
\end{equation} 

\begin{equation}
1 + z = {a \over a_0}
\end{equation} 

\section{The evolution of the universe}

Assuming that the matter-energy content of the universe can be described as a perfect fluid, which follows the state equation $P = \omega \rho$, the gravitational field equation gives us the temporal evolution $a(t)$ of the FLRW metric in differential form.

Friedman equation
\begin{equation}
H^2 \equiv \left({\dot{a} \over a}\right)^2 = H_0^2\sum_\omega \left({a_0 \over a}\right)^3(1+w)
\end{equation} 

Hubble constant $H_0 \sim 70 (km/s)/Mpc $

\begin{equation}
H_\Lambda(z) = H_0 \sqrt{\Omega_M(1+z)^3 + \Omega_\Lambda}
\end{equation}

$\Omega_M + \Omega_\Lambda = 1$

Dark Energy relative density $\Omega_\Lambda \sim 0.73$

Mater relative density $\Omega_M \sim 0.26$

\section{Luminosity and Angular distances}

Comoving distance 
\begin{equation}
r \equiv {c \over H_0}\int_0^{z} {dz \over H(z)}
\end{equation}

Luminosity distance 
\begin{equation}
D_L \equiv \sqrt{L \over 4 \pi F}  = r (1+z)
\end{equation}

Angular distance 
\begin{equation}
D_A \equiv {\ell \over \theta} = {r \over (1+z)}
\end{equation}

\section{Photometry}

\subsection{Flux and Aparent magnitude}
Aparent magnitude
\begin{equation}
m - m_0 \equiv -2.5\log{F \over F_0}
\end{equation}

$F \equiv \int^{\infty}_0 f(\nu)R(\nu){d\nu \over \nu} = \int^{\infty}_0 f(\lambda)R(\lambda)\lambda {d\lambda \over c}$ 
\citep{Hogg1996}

$\nu \lambda = c \Longrightarrow {d\nu \over \nu} = -{d\lambda \over \lambda}$

$f(\nu)d\nu =f(\lambda)d\lambda \Longrightarrow \nu f(\nu) = \lambda f(\lambda)$

$[F] = [f(\nu)] = \left[{ Flux \over time^{-1}} \right]$, $[f(\lambda)] = \left[{ Flux \over longitude} \right]$

\subsection{The AB Systems}

The AB system \citep{Oke1970} 

$m_0 = 0$

$F_0$ such that $f^{AB}_{0}(\nu) = 3631Jy$

$1Jy = 10^{-23}erg \cdot cm^{-2} \cdot s^{-1} \cdot Hz^{-1} = 1.51 \cdot 10^7 \cdot photons \cdot m^{-2} \cdot s^{-1} \cdot dlog^{-1}\lambda$

$m_{AB} \equiv -2.5 \log \left({F \over F_0}\right) \Longleftrightarrow F = F_0 10^{-0.4m_{AB}}$

$m_{AB}(\nu) = -2.5 \log f_{\nu}(\nu) - 48.6$ \citep{Oke1982}

%$m_{AB}(\lambda) = -2.5 \log f(\lambda) - 21.1$

%$1Jy = 10^{-12}erg \cdot cm^{-2} \cdot s^{-1} \cdot \AA^{-1} = 1.51 \cdot 10^7 \cdot photons \cdot m^{-2} \cdot s^{-1} \cdot dlog^{-1}\lambda$

\subsection{Absolute magnitude and K-correction}

Absolute magnitude 
\begin{equation}
M = m - DM - K
\end{equation}

Distance Modulus 
\begin{equation}
DM \equiv 5 \log \left[{D_L \over 10pc} \right]
\end{equation}

K-correction 
\begin{equation}
K \equiv -2.5 \log \left[{1 \over 1+z}{\int^{\infty}_{0} f\left( \lambda / 1+z \right)R(\lambda)\lambda d\lambda \over f(\lambda)R(\lambda)\lambda d\lambda}\right]
\end{equation}

\subsection{Luminosity function}

Schechter function 

\begin{equation}
n(x) \equiv {dN \over dx} = \phi^* x^a e^{-x} 
\end{equation}

where $x \equiv {M \over M^*}$

\subsection{Measuring Magnitudes}

Model magnitudes: identical apertures

Apper magnitudes

Auto, psf 

fibre

De Vaucouleurs

Petrosian

\section{Photometric Redshifts}

\subsection{Template Fitting Methods}

\begin{equation}
\chi^2(z,t)=\sum_i {(m_i -  m_i^t(z))^2 \over  \sigma_m^2}
\end{equation}

%ANNz \cite{Collister:2003fx}

%ArborZ \cite{Gerdes:2010kr}

%\begin{figure}[H]
%\centering
%\includegraphics[width=150mm]{../Data/Plot/CE_NEW.pdf}
%\caption{An example figure.}
%\label{templates}
%\end{figure}

\subsection{Bayesian statistics} 

\textbf{BPZ} \citep{Benitez2000}

Probability density distribution
\begin{equation}
p(z \mid m_i, m_{ref}) \propto \sum_t \Pi(z,t \mid m_{ref})L(m_i \mid z,t)
\end{equation}

Likelihood $L(m_i \mid z, t) \propto \exp \chi^2 (z,t)$

Prior
\begin{equation}
\Pi(z, t \mid m) \propto f_t e^{-k_t(m-m_0)} \cdot z^{\alpha_t}\exp \left\lbrace -\left[ {z \over z_{mt}(m)} \right]^{\alpha_t} \right\rbrace
\end{equation}

$z_{mt}(m) = z_{0t} + k_{mt}(m-m_0)$

%\begin{figure}[H]
%\includegraphics[type=pdf,ext=.pdf,read=.pdf, width=185mm]{/Users/polstein/Dropbox/Tesi/photoz/mock.r260.n1e6.s10.121027_42NB.100A/BPZ/PRIOR/plots/pau_prior_p_z_t_22.0}
%\caption{An example figure.}
%\label{bs_pz_results}
%\end{figure}

\subsection{Quality cuts}

\begin{equation}
odds \equiv \int^{z(phot) + \delta z}_{z(phot) - \delta z} p(z \mid m_i, m_{ref})dz
\end{equation}

\section{Baryonic Acoustic Oscilations}
\begin{equation}
r_{BAO} \sim {c_s \over H_0} \int_{1100}^{\infty} {dz \over H_\Lambda(z)} \sim 150 Mpc
\end{equation}

where $c_s \sim c \sqrt{1\over3}$

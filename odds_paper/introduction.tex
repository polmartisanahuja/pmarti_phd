Most if not all photo-z algorithms provide not only a best estimate for the galaxy redshift but also an estimate of the quality of its determination, be it simply an error estimatation or something more sophisticated like the {\it odds} parameter in the {\tt BPZ} code~\citep{Benitez2000}. Applying cuts on the value of this quality parameter, one can clean up the sample from galaxies with an unreliable photo-z determination~\citep{Benitez2000}, or even select a smaller sample of galaxies with significantly higher photo-z precision~(\paperpau).

However, we will show in this \doctype\ that these quality cuts can affect very significantly the observed clustering of galaxies in the sample retained after the cuts, thereby biasing the cosmological information that can be obtained. Therefore, this effect needs to be corrected. Fortunately, this can be readily achieved using a technique similar to that presented in~\citet{Ho2012}, to deal with, among others, the effect of stars contaminating a galaxy sample. 

The outline of this \doctype\ is as follows. Section~\ref{sec:data} discusses the galaxy samples that we use in our study: the Mega-Z photometric galaxy sample ~\citep{Collister2007} and its companion, the 2SLAQ spectroscopic sample ~\citep{Cannon2006}, which we use to characterize the redshift distribution of the galaxies in Mega-Z. We will also describe the photo-z algorithm used ({\tt BPZ}) and the resulting redshift distributions in four photometric redshift bins in Mega-Z. In section~\ref{sec:clustering} we present the measurement of the galaxy-galaxy angular correlations within, and cross-correlations across, the four photo-z bins in Mega-Z, comparisons with the theoretical expectations, and the effects of applying several photo-z quality cuts to the data. Section~\ref{sec:correction} introduces the correction we have devised for the effect of the photo-z quality cut and applies it to the data, resulting in corrected angular correlation functions that we then compare with the predictions. In section~\ref{sec:BAO} we extract the Baryon Acoustic Oscilation (BAO) angular scale from the corrected data and compare it with the result obtained without any photo-z quality cuts. Finally, in section~\ref{sec:discussion} we discuss the relevance of our results, particularly for previous studies that applied photo-z quality cuts while ignoring their effects on clustering, and we offer some conclusions.

\section{Discussion and Conclusions}
\label{sec:discussion}
%
In the previous sections we have seen how photo-z
quality cuts, if left uncorrected, can severely bias the measured
galaxy angular auto- and cross-correlations. The effect, as seen in
Figs.~\ref{auto_odcorr} and~\ref{cross_odcorr}, consists mostly of a
large increase in the correlation across the whole range in angular
separation, although slightly more prominent at larger
separations. This is not unlike the effect reported in other clustering
studies based on very similar samples, such as those in~\citet{Thomas2011a}
and in~\citet{Crocce2011}. In those papers an excess of clustering has
been observed in the photometric redshift bin $0.6 < z < 0.65$. In at
least one of the papers~\citep{Crocce2011} a photo-z quality cut is
perfomed, eliminating about 16\% of the galaxies, but no attempt is
made to correct for the possible effect of this cut on the measured
correlations.  While a quick look at Fig.~\ref{auto_odcorr} reveals
that in our case the effect of the photo-z quality cut is more
prominent at lower redshifts, the issue may deserve more thorough study, 
which lies beyond the scope of this thesis.

It is intriguing to see that, at least in the $0.45 < z < 0.50$
photo-z bin, there is a correlation between galaxy density and 
{\em odds} value even before any cut on the value of the {\em odds}
(bottom-left plot in Fig.~\ref{gal_map}). This
correlation then leads to extra galaxy auto- and cross-correlations
whenever that first photo-z bin is involved (top-left plot in
Fig.~\ref{auto_odcorr} and top-right plot in Fig.~\ref{cross_odcorr}). 
The correction method we propose
eliminates this extra correlation very effectively (see the corresponding plots in
Figs.~\ref{auto_odcorr} and \ref{cross_odcorr}), but the question
remains: what is it that we are actually eliminating? Or: where does
this galaxy-{\em odds} correlation come from? 

One possibility is that
it comes from systematic effects in the survey that are otherwise
uncorrected: differences in seeing conditions, airmass, extinction,
etc. between different areas of the survey will lead to correlated differences in
galaxy density and in the value of the {\em odds} parameter. In
general, these systematic effects would result in an additive extra
correlation. In this case, correcting for this
spurious correlation will mitigate the effects of those systematic issues.

However, in general, another possibility is that the galaxy-{\em odds} correlation is due to the
fact that different galaxy types have different clustering amplitudes
(different biases) and, at the same time, also have different mean
photo-z precisions, hence different mean {\em odds}. 
%For instance, LRGs have typically larger bias and more precise photo-z
%(hence larger {\em odds}) than spiral galaxies. 
In this case, the {\em odds} correction could be removing genuine
galaxy-galaxy correlations. Alternatively, one could interpret that
the correction would be changing the average galaxy
type (and hence bias) of the sample. Hence, this would result in a
multiplicative extra correlation.

In our case, since the sample is largely composed of LRGs, the
galaxy-{\em odds}  correlation we observe in the lower photo-z bin is likely 
due to the uncorrected systematic effects mentioned above, and,
therefore, it makes sense to apply the {\em odds} correction even without an
explicit {\em odds} cut. We observe that, indeed, the extra
correlation we observe seems to be additive in nature (i.e.~roughly
constant as a function of angular scale).
Figures~\ref{auto_odcorr} and \ref{cross_odcorr} show that, even with
no {\em odds} cut, the agreement
with the predictions improves once the {\em odds} correction
has been applied.

\vspace*{1em}

In summary, using the Mega-Z DR7 galaxy sample and the {\tt BPZ}
photometric redshift code,
we have shown that applying moderate galaxy photo-z quality cuts
may lead to large biases in the measured galaxy auto- and cross-correlations.
However, a correction method
derived within the framework presented in~\citet{Ho2012, Ross2011}
%{\color{red} Add Ross et al. 2012.} 
manages to recover the original correlation functions and,
in particular, does not bias the extraction of the BAO peak. It
remains to be seen whether this correction might eliminate some or
all of the excess correlation observed by several groups in galaxy
samples essentially identical to the one we use.

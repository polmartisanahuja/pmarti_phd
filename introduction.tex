Since ancient times humankind has wondered about the nature of the night sky and its celestial objects, to which it always has attributed a mystical or religious character. However, with the advent of the Modern era, and in turn the scientific method, we gradually realized that the laws of nature that govern everyday phenomena also explain the celestial world. Stars are objects just like the Sun but at much greater distance, and planets are worlds like Earth, also trapped in the Sun's gravitational field. Early last century, it was thought that the Universe was basically composed of stars uniformly spread out over space, which had occupied the same place since ever. Later, astronomers realized that in fact stars were grouped into large islands called galaxies and these again were grouped forming galaxy clusters and superclusters placed between huge voids. This is known as the large-scale structure of the Universe. Moreover, these galaxies seemed to be moving away from each other the further the faster, as if the space between them was growing. This led to the undeniable conclusion that the Universe is expanding, and it had not always been like now. In fact, about 13,700 million years ago, everything should have been confined in a much smaller space that eventually expanded and evolved into the Universe we know today. This is referred to as the theory of the Big Bang, which gave birth to a new branch of Astronomy called Cosmology, studying the matter-energy content and evolution of the Universe as a whole. A brief introduction to the astronomical and cosmological concepts necessary to understand this work is presented in the first half of chapter~\ref{ch:framework}.

Einstein's gravitational field equations predict that the evolution of the Universe and the growth of the large-scale structure are directly related to its matter-energy content. Nowadays, by measuring the evolution of the 3D position of a large number of distant galaxies, together with other cosmological probes, cosmologists have determined that only about five percent of the content of the universe is ordinary matter (molecules, atoms, plasma, neutrinos and photons) while the rest should be in two types of matter-energy whose origin and properties are unknown: Dark Matter and Dark Energy. This presents a challenge for modern Cosmology. There are several theories that attempt to respond to it, however many more galaxy surveys are needed to reach the necessary precision to rule out or confirm some of them. Spectroscopic surveys~(2dF, \citet{Colless2001}; VVDS, \citet{LeFevre2005}; WiggleZ, \citet{Drinkwater2010}; BOSS, \citet{Dawson2013}) provide a 3D image of the galaxy distribution in the near universe, but most of them suffer from limited depth, incompleteness and selection effects. Imaging surveys~(SDSS, \citet{York2000}; PanSTARRS, \citet{kaiser2000}; LSST, \citet{Tyson2003}; DES) solve these problems but, on the other hand, do not provide a true 3D picture of the universe, due to their limited resolution in the position along the line of sight, which is obtained measuring the galaxy redshift through photometric techniques using a set of broad-band filters, a technique known as photometric redshift (photo-$z$). The Physics of the Accelerated Universe (PAU) survey at the William Herschel Telescope (WHT) in the Roque de los Muchachos Observatory (ORM) in the Canary island of La Palma (Spain) will use narrow-band filters to try to achieve a quasi-spectroscopic precision in the redshift determination that will allow it to map the large-scale structure of the universe in 3D using photometric techniques, overcoming the limitations of spectroscopic surveys~\citep{Benitez2009}. A detailed explanation of photo-$z$ methods and an overview of the galaxy redshift surveys used in this work are given in the second half of chapter~\ref{ch:framework}.

The photo-$z$ performance capabilities of a narrow-band filter system, like that of the PAU@WHT survey, constitute the main topic of this thesis and are detailed in chapter~\ref{ch:pau}. The stringent photo-$z$ precision requirements of this survey, defined in \citet{Gaztanaga2012}, can only be achieved by applying photo-$z$ quality cuts, that is, by not considering galaxies with unreliable photo-$z$ estimations. However, if these galaxies are not removed homogeneously over the sky, this can severely bias the large-scale structure measurements. Since photo-$z$s depend mainly on the photometry whose quality depends on the atmospheric conditions of the observed region of the sky, the distribution of the photo-$z$ quality over it will not be homogeneous. Therefore, one must be careful when applying this kind of cuts. In chapter~\ref{ch:odds} we study the impact of these cuts on the angular clustering of a real luminous red galaxy based catalog \citep{Collister2007} from the Sloan Digital Sky Survey (SDSS) and propose a method to correct for it. 

Finally, in chapter~\ref{ch:conclusions}, we conclude the thesis with a brief summary of the two previous chapters, how they relate to each other, and providing a first look at their implications and possible future studies. 

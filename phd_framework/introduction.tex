In this chapter we introduce the concepts and tools necessary to follow the next two chapters. The outline is as follows. In section~\ref{sec:cosmo_model} we present the $\Lambda$CDM cosmological model, the main probes supporting it, the concepts of cosmological redshift and distances, and a brief introduction to the inhomogeneous Universe. In section~\ref{sec:photometry} we give basic notions underlying the science of photometry such as: fluxes, apparent and absolute magnitudes, photometric systems, etc. In section~\ref{sec:theo_photoz} we introduce the concept of photometric redshift, the main topic of this thesis, we explain its origin and usefulness on observational cosmology and enumerate different methods to compute it, emphasizing the BPZ method \citep{Benitez2000} used in this thesis. Finally, in section \ref{sec:surveys} we present the three galaxy redshift surveys whose data are used in this thesis, when it is available (SDSS and 2dFGRS), or simulated, when it is not (PAU@WHT Survey).
